% ==============================================================================
% EE413A
% Lab 2 - The Operational Amplifier
% ---------------------------------
% Last updated 2014-12-03
%
% Author:
% Jonas Sjöberg     <tel12jsg@student.hig.se>
% Esther Hedlund    <tfk13ehd@student.hig.se> 
%
% License:
% Creative Commons Attribution-NonCommercial-ShareAlike 4.0 International
% See LICENSE.md for full licensing information.
% ==============================================================================


% ==============================================================================
% INCLUDES AND CONFIGURATION
% ==============================================================================
\documentclass[11pt,a4paper]{article}

\usepackage[utf8]{inputenc}
\usepackage{siunitx} % Provides the \SI{}{} and \si{} command for typesetting SI
\usepackage{amssymb}
\usepackage{amsmath}
\usepackage{amsfonts}
\usepackage{graphicx}
\usepackage{longtable,booktabs}
\usepackage{microtype}

\setlength\parindent{0pt} % Removes all indentation from paragraphs

% ==============================================================================
% DOCUMENT METADATA 
% ==============================================================================
\title{EE413 \\ Lab 005 \\ the Operational Amplifier}
%\author{{Jonas Sjöberg} \and {Esther Hedlund}}

\author{\\
  Jonas Sjöberg\\
  Högskolan i Gävle,\\
  Elektronikingenjörsprogrammet,\\
  \texttt{tel12jsg@student.hig.se}\\
  \\
  Esther Hedlund\\
  Högskolan i Gävle,\\
  Elektronikingenjörsprogrammet,\\
  \texttt{tfk13ehd@student.hig.se}\\}

\date{}
% ==============================================================================
\begin{document}
% ==============================================================================
\maketitle

\begin{center}
\begin{tabular}{l r}
    Data Performed: & 26 November 2014 \\
    Instructor: & Nauman Masud, Shoaib Amin
\end{tabular}
\end{center}

% ==============================================================================
% ABSTRACT
% ==============================================================================
\begin{abstract}
This lab is meant to show the practical use of the operational amplifier in
analog circuit design. Several common circuit configurations will be discussed.
\end{abstract}

\newpage

{
%\hypersetup{linkcolor=black}
\setcounter{tocdepth}{3}
\tableofcontents
}

\newpage

% ==============================================================================
% SECTION: Circuit prototyping setup
% ==============================================================================
\section{Circuit prototyping setup}\label{setup}
The circuit was build on a solderless breadboard, using through-hole parts.
A classic 741 op amp was used with a +/-15V power supply.
No decoupling caps was used and signal lines were not properly terminated.

For measurements the following instruments was used; HP34401A bench multimeter,
HP33120A signal generator and Agilent E3631A lab power supply.

% ==============================================================================
% SECTION: Inverting DC Amplifier
% ==============================================================================
\section{Inverting DC Amplifier}\label{inverting-dc-amplifier}
% ==============================================================================

\subsection{Theory}\label{invDC-theory}
% ==============================================================================

\begin{figure}[htbp]
    \centering
        \includegraphics[scale=0.5]{img/invDCamp.png}
    \caption{Inverting DC amplifier}
    \label{fig:invDCamp}
\end{figure}

The basic topology for an inverting amplifier is shown in
Figure~\ref{fig:invDCamp}.  Gain $A_v$, can be expressed as a ratio of the
feedback impedance to the input impedance. Op amp action makes the negative
input appear as a "virtual earth" summing node. The voltage drops across the
resistors scale linearly with their value, and since the op amp compensates to
ensure equality in the summing junction, the net effect is an amplified and
inverted output.

\begin{equation}
    A_v = \frac{R_2}{R_1}
\end{equation}

The circuit gain for ideal components is therefore;

For $R_2 = 100k\Omega$:

\begin{align} 
A_v     &= \frac{V_{out}}{V_{in}} = -\frac{R_2}{R_1}\\
        &= \frac{100k\Omega}{10k\Omega} = 10\times\\
        &= 20 \times \log{\frac{10}{1}} = 20dB  
\end{align}

For $R_2 = 10k\Omega$:

\begin{align} 
A_v     &= \frac{V_{out}}{V_{in}} = -\frac{R_2}{R_1}\\
        &= \frac{10k\Omega}{10k\Omega} = 10\times\\
        &= 20 \times \log{\frac{1}{1}} = 0dB  
\end{align}

In both cases, the signal phase is inverted $180^\circ$.


\subsection{Measurements}\label{invDC-measurements}
% ==============================================================================

Measured values for the test setup are shown in Table~\ref{invDCtable1} and
Table~\ref{invDCtable2}.

\begin{longtable}[c]{@{}lll@{}}
\toprule\addlinespace
$U_{in}$ (V) & $U_{out}$ (V) & $Av$ ($\times$)
\\\addlinespace
\midrule\endhead
-0.103 & +1.087  & -10.35
\\\addlinespace
-1.008  & +10.236   & -10.15
\\\addlinespace
+1.004  & -10.104   & -10.06
\\\addlinespace
\bottomrule
\addlinespace
\caption{$R_2 = 100k\Omega$}
\label{invDCtable1}
\end{longtable}

\begin{longtable}[c]{@{}lll@{}}
\toprule\addlinespace
$U_{in}$ (V) & $U_{out}$ (V) & $Av$ ($\times$)
\\\addlinespace
\midrule\endhead
-0.1003 & +0.112  & -1.116
\\\addlinespace
-1.000  & +1.038   & -1.038
\\\addlinespace
+1.005  & -1.027   & -1.022
\\\addlinespace
\bottomrule
\addlinespace
\caption{$R_2 = 10k\Omega$}
\label{invDCtable2}
\end{longtable}

% Circuit #1
For $R_2 = 100k\Omega$, the actual measured in circuit values of $R_2$ and
$R_1$ were 119k$\Omega$ and 11.7k$\Omega$, respectively. Calculated circuit
gain for non-ideal, real components;

\begin{align} 
A_v     &= \frac{V_{out}}{V_{in}} = -\frac{R_2}{R_1}\\
        &= \frac{119k\Omega}{11.7k\Omega} = 10.17\times\\
        &= 20 \times \log{\frac{\frac{119k\Omega}{11.7k\Omega}}{1}} = 20.15dB  
\end{align}


% Circuit #2
For $R_2 = 10k\Omega$, the actual measured in circuit values of $R_2$ and $R_1$
were 12.17k$\Omega$ and 11.7k$\Omega$, respectively. Calculated circuit gain
for non-ideal, real components;

\begin{align} 
A_v     &= \frac{V_{out}}{V_{in}} = -\frac{R_2}{R_1}\\
        &= \frac{12.17k\Omega}{11.7k\Omega} = 1.04\times\\
        &= 20 \times \log{\frac{\frac{12.17k\Omega}{11.7k\Omega}}{1}} = 0.34dB  
\end{align}

% ==============================================================================
% SECTION: Inverting AC Amplifier
% ==============================================================================
\section{Inverting AC Amplifier}\label{inverting-ac-amplifier}
% ==============================================================================

\subsection{Theory}\label{invAC-theory}
% ==============================================================================

\begin{figure}[htbp]
    \centering
    \includegraphics[scale=0.5]{img/invACamp.png}
    \caption{Inverting AC amplifier}
    \label{fig:invACamp}
\end{figure}


The basic topology for an inverting AC amplifier is shown in
Figure~\ref{fig:invACamp}.


\begin{equation}
    A_v = 1+\frac{R_2}{R_1}
\end{equation}

The circuit gain for ideal components is;

For $R_2 = 100k\Omega$:

\begin{align} 
A_v     &= \frac{V_{out}}{V_{in}} = -\frac{R_2}{R_1}\\
        &= \frac{100k\Omega}{10k\Omega} = 10\times\\
        &= 20 \times \log{\frac{10}{1}} = 20dB  
\end{align}

For $R_2 = 10k\Omega$:

\begin{align} 
A_v     &= \frac{V_{out}}{V_{in}} = -\frac{R_2}{R_1}\\
        &= \frac{10k\Omega}{10k\Omega} = 10\times\\
        &= 20 \times \log{\frac{1}{1}} = 0dB  
\end{align}

In both cases, the signal phase is inverted $180^\circ$.


\subsection{Measurements}\label{invAC-measurements-1}
% ==============================================================================


\subsection{Oscilloscope shots}\label{invAC-oscilloscope-shots}
% ==============================================================================
Oscilloscope photos in figure~\ref{fig:invACamp20dB_scope} and
figure~\ref{fig:invACampunity_scope} show the amplifier input on channel one
the amplifier output on channel two.  Channel two volts/div is set to
compensate for the high impedance 10:1 probe setting. 

\begin{figure}[htbp]
    \centering
    \includegraphics[width=0.75\textwidth]{img/invACamp-x10.jpg}
    \caption{Inverting AC amplifier - 20dB gain}
    \label{fig:invACamp20dB_scope}
\end{figure}

\begin{figure}[htbp]
    \centering
    \includegraphics[width=0.75\textwidth]{img/invACamp-x1.jpg}
    \caption{Inverting AC amplifier - unity gain}
    \label{fig:invACampunity_scope}
\end{figure}


% ==============================================================================
% SECTION: Non-inverting DC Amplifier
% ==============================================================================
\section{Non-inverting DC Amplifier}\label{non-inverting-dc-amplifier}
% ==============================================================================

\subsection{Theory}\label{noninvDC-theory}
% ==============================================================================

\begin{figure}[htbp]
    \centering
    \includegraphics[scale=0.5]{img/noninvDCamp.png}
    \caption{Non-inverting DC amplifier}
    \label{fig:noninvDCamp}
\end{figure}

The basic topology for an non-inverting DC amplifier is shown in
Figure~\ref{fig:noninvDCamp}.  Gain $A_v$, is set by the attenuation-factor of
the circuit in the feedback-loop.  A fraction of the output is fed back,
causing the op amp to compensate and in effect amplify.

\begin{equation}
    A_v = 1+\frac{R_2}{R_1}
\end{equation}


\subsection{Measurements}\label{measurements-2}
% ==============================================================================

Measured values for the test setup are shown in Table~\ref{noninvDCtable1} and
Table~\ref{noninvDCtable2}.

\begin{longtable}[c]{@{}lll@{}}
\toprule\addlinespace
$U_{in}$ (V) & $U_{out}$ (V) & $Av$ ($\times$)
\\\addlinespace
\midrule\endhead
+0.1007 & +0.2164 2 & .15
\\\addlinespace
+1.002 & +2.048 2 & .04
\\\addlinespace
-1.005 & -2.03 2 & .019
\\\addlinespace
\bottomrule
\addlinespace
\caption{$R_2 = 10k\Omega$}
\label{noninvDCtable1}
\end{longtable}

\begin{longtable}[c]{@{}lll@{}}
\toprule\addlinespace
$U_{in}$ (V) & $U_{out}$ (V) & $Av$ ($\times$)
\\\addlinespace
\midrule\endhead
+0.1009 & +1.178 & 11.67
\\\addlinespace
+1.1013 & +11.3 & 11.15
\\\addlinespace
-1.004 & -11.09 & 11.05
\\\addlinespace
\bottomrule
\addlinespace
\caption{$R_2 = 100k\Omega$}
\label{noninvDCtable2}
\end{longtable}


% ==============================================================================
% SECTION: Non-inverting AC Amplifier
% ==============================================================================
\section{Non-inverting AC Amplifier}\label{non-inverting-ac-amplifier}
% ==============================================================================

\subsection{Theory}\label{noninvAC-theory}
% ==============================================================================
The lab circuit for the non-inverting AC amplifier is identical to the
non-inverting DC amplifier. The basic topology is shown in
figure~\ref{fig:noninvACamp}.

\begin{figure}[htbp]
    \centering
    \includegraphics[scale=0.5]{img/noninvACamp.png}
    \caption{Non-inverting AC amplifier}
    \label{fig:noninvACamp}
\end{figure}


\subsection{Measurements}\label{measurements-3}
% ==============================================================================



\subsection{Oscilloscope shots}\label{noninvAC-oscilloscope-shots}
% ==============================================================================
Oscilloscope photos in figure~\ref{fig:noninvACamp-x1_scope} and
figure~\ref{fig:noninvACamp-x10_scope} show the amplifier input on channel one
the amplifier output on channel two.  Channel volts/div is set to compensate
for the high impedance 10:1 probe setting. 

\begin{figure}[htbp]
    \centering
    \includegraphics[width=0.75\textwidth]{img/noninvACamp-x1.jpg}
    \caption{Non-inverting AC amplifier - 6dB gain}
    \label{fig:noninvACamp-x1_scope}
\end{figure}

\begin{figure}[htbp]
    \centering
    \includegraphics[width=0.75\textwidth]{img/noninvACamp-x10.jpg}
    \caption{Non-inverting AC amplifier - 20dB gain}
    \label{fig:noninvACamp-x10_scope}
\end{figure}



% ==============================================================================
% SECTION: Active full wave rectifier
% ==============================================================================
\section{Active full wave rectifier}\label{active-full-wave-rectifier}
% ==============================================================================

\subsection{Theory}\label{fwr-theory}
% ==============================================================================
Compared to a passive rectifier circuit, the active rectifier does not suffer
from the ``deadzone'' when the signal is too small to turn on the rectifying
diode. The op amp compensates for the diode forward voltage drop.  The circuit
output is a full wave rectified version of the signal, with a frequency limit
mostly set by the op amp bandwidth. Diode D2 prevents the op amp from hitting
the rail hard when D1 is reverse biased. This makes the recovery and rise time
faster when D1 biases on and this in turn improves circuit response times.

\begin{figure}[htbp]
    \centering
    \includegraphics[scale=0.5]{img/fwr.png}
    \caption{Active full wave rectifier}
    \label{fig:fwr-schem}
\end{figure}


\subsection{Oscilloscope shots}\label{fwr-oscilloscope-shots}
% ==============================================================================
Oscilloscope photos in Figure~\ref{fig:fwr1_scope} and
Figure~\ref{fig:fwr2_scope} show the input signal on channel one and the full
wave rectifier output on channel two.  Channel volts/div is set to compensate
for the high impedance 10:1 probe setting. 

\begin{figure}[htbp]
    \centering
    \includegraphics[width=0.75\textwidth]{img/fwr1.jpg}
    \caption{Full wave rectifier - input and rectified output}
    \label{fig:fwr1_scope}
\end{figure}

\begin{figure}[htbp]
    \centering
    \includegraphics[width=0.75\textwidth]{img/fwr2.jpg}
    \caption{Full wave rectifier - X/Y-view}
    \label{fig:fwr2_scope}
\end{figure}


% ==============================================================================
% SECTION: Low pass filter
% ==============================================================================
\section{Low pass filter}\label{lowpass-filter}
% ==============================================================================

\subsection{Theory}\label{lpf-theory}
% ==============================================================================
\begin{figure}[htbp]
    \centering
        \includegraphics[scale=0.5]{img/lowpassfilter.png}
    \caption{Low pass filter}
    \label{fig:lpf-schem}
\end{figure}

The basic topology for a first order active low pass filter is shown in Figure~\ref{fig:lpf-schem}.
Gain $A_v$, can be expressed as a ratio of the
feedback impedance to the input impedance. 

\begin{equation}
    A_v = \frac{Z_f}{Z_i}
\end{equation}

The circuit transfer function is;

\begin{align} 
A_v     &= \frac{V_{out}}{V_{in}} = -\frac{Z_f}{Z_i}\\
        &= \frac{R_2}{R_1} \times \frac{1}{\sqrt{1+\left(\omega R_2 C\right)^2}}\\
        &= \frac{33k\Omega}{3.3k\Omega} \times \frac{1}{\sqrt{1+\left(\omega \times 33k\Omega \times 2.2nF\right)^2}}
\end{align}


\subsection{Measurements}\label{lpf-measurements}
% ==============================================================================
The measured frequency response is shown as a bode plot in Figure~\ref{fig:lpf-magnitude} and the phase is shown in Figure~\ref{fig:lpf-phase}.

\begin{figure}[htbp]
    \centering
        \includegraphics[]{img/lpf-magnitude.eps}
    \caption{Low pass filter frequency response}
    \label{fig:lpf-magnitude}
\end{figure}

\begin{figure}[htbp]
    \centering
        \includegraphics[]{img/lpf-phase.eps}
    \caption{Low pass filter phase response}
    \label{fig:lpf-phase}
\end{figure}


\subsection{Oscilloscope shots}\label{lpf-oscilloscope-shots}
% ==============================================================================
Oscilloscope photos in figure~\ref{fig:lpf-5k} show the low pass filter input and
output with a 5kHz input signal.
Figure~\ref{fig:lpf-10k} shows the low pass filter input and output with a 10kHz input signal. Channel two volts/div is set to compensate for the high impedance 10:1 probe setting. 

\begin{figure}[htbp]
    \centering
    \includegraphics[width=0.75\textwidth]{img/lpf-5k.jpg}
    \caption{Low pass filter - 5kHz input signal}
    \label{fig:lpf-5k}
\end{figure}

\begin{figure}[htbp]
    \centering
    \includegraphics[width=0.75\textwidth]{img/lpf-10k.jpg}
    \caption{Low pass filter - 10kHz input signal}
    \label{fig:lpf-10k}
\end{figure}


% ==============================================================================
% SECTION: Zener clipper
% ==============================================================================
\section{Zener diode clipper}\label{zener-clipper}
% ==============================================================================

\subsection{Theory}\label{clipper-theory}
% ==============================================================================
\begin{figure}[htbp]
    \centering
        \includegraphics[scale=0.5]{img/zenerclipper.png}
    \caption{Zener diode clipper}
    \label{fig:clipper-schem}
\end{figure}

The schematic for a inverting amplifier with a diode amplitude limiting clipper
is shown in Figure~\ref{fig:clipper-schem}.

\subsection{Measurements}\label{clipper-measurements}
% ==============================================================================
Circuit functionality was tested by increasing the signal amplitude until
limiting occured.

\subsection{Oscilloscope shots}\label{clipper-oscilloscope-shots}
% ==============================================================================
Oscilloscope photos in figure~\ref{fig:clip_xy_1} to figure~\ref{fig:clip_xy_7}
shows the transfer function of the zener clipper as an X/Y-graph.
The curvature of the graph is the zener diodes gradually turning on as the
input signal voltage rises to their conducting voltage.

\begin{figure}[htbp]
    \centering
    \includegraphics[width=0.75\textwidth]{img/clip_xy_1.jpg}
    \caption{Zener diode clipper photo 1}
    \label{fig:clip_xy_1}
\end{figure}

\begin{figure}[htbp]
    \centering
    \includegraphics[width=0.75\textwidth]{img/clip_xy_2.jpg}
    \caption{Zener diode clipper photo 2}
    \label{fig:clip_xy_2}
\end{figure}

\begin{figure}[htbp]
    \centering
    \includegraphics[width=0.75\textwidth]{img/clip_xy_3.jpg}
    \caption{Zener diode clipper photo 3}
    \label{fig:clip_xy_3}
\end{figure}

\begin{figure}[htbp]
    \centering
    \includegraphics[width=0.75\textwidth]{img/clip_xy_4.jpg}
    \caption{Zener diode clipper photo 4}
    \label{fig:clip_xy_4}
\end{figure}

\begin{figure}[htbp]
    \centering
    \includegraphics[width=0.75\textwidth]{img/clip_xy_5.jpg}
    \caption{Zener diode clipper photo 5}
    \label{fig:clip_xy_5}
\end{figure}

\begin{figure}[htbp]
    \centering
    \includegraphics[width=0.75\textwidth]{img/clip_xy_6.jpg}
    \caption{Zener diode clipper photo 6}
    \label{fig:clip_xy_6}
\end{figure}

\begin{figure}[htbp]
    \centering
    \includegraphics[width=0.75\textwidth]{img/clip_xy_7.jpg}
    \caption{Zener diode clipper photo 7}
    \label{fig:clip_xy_7}
\end{figure}


% ==============================================================================
% SECTION: Results
% ==============================================================================
\section{Results}\label{setup}
% ==============================================================================
The measurement errors and differences between the theoretical calculations using
ideal components is to be expected. With 5% tolerance resistors, low-spec op amp
and a crude prototyping setup, parasitics play a major role in the behavior of the
circuit. 

\newpage

% ==============================================================================
% SECTION: References
% ==============================================================================
\section{References}\label{references}

\subsection{Literature}\label{literature}
% ==============================================================================
Horowitz and Hill - The Art of Electronics, Cambridge University Press
1989. Horowitz and Hayes - Student Manual for the Art of Electronics,
Cambridge 1989.

\subsection{Sources}\label{sources}
% ==============================================================================
Full source, including spice simulation files, CSV data, schematics, etc
is available at https://github.com/jonasjberg/EE413-lab02


% ==============================================================================
\end{document}
% ==============================================================================